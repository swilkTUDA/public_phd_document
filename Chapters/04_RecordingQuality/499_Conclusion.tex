% !TeX spellcheck = en_US
\section{Conclusion}
\label{sec_499_conclusion}
This chapter introduces the first quantified models on the impact of camera shaking, harmful occlusions, and camera misalignment on the perceived quality of videos. 
It is shown that camera shakes have the highest impact on the perceived quality; harmful occlusions reduce the quality nearly as much as camera shakes, but only if they occur in the \ac{RoI}. 
In contrast, camera misalignments are perceived as less disturbing. 
Besides the impact of the degradation, individual characteristics are also assessed, such as the duration and speed of a camera shake.
Depending on the video genre as well as degradation type, different characteristics have a degrading impact on the perceived quality.

Also, video composition applications can leverage different video views being recorded in parallel.
These views differ regarding the device's position in relation to the recorded scene.
In crowdsourced subjective studies with several hundred of workers, quality models are created for different recording positions and video genres.
The models indicate that an increasing distance to a scene degrades the perceived quality, whereas the relative angle to the scene plays only a minor role.

The proposed models are used for the creation of automatic quality assessment algorithms, which are discussed in Chapter~\ref{chapter:550_scalable_quality_assessment}, and for the video composition algorithm introduced in Chapter~\ref{chapter:600_videocomposition}.