% !TeX spellcheck = en_US
The first contribution of this thesis is the analysis of quality-degrading artifacts in \ac{UGV}, which are related to a recording person's limited skills or a lack of suitable equipment such as a tripod.
Assessing the impact of these degradations on human perception leads to a subtype of the perceived quality, the recording quality.
In detail, the impact of the degradations of camera shake, harmful occlusions and camera misalignment on the perceived quality are assessed.
Different characteristics of each degradation, e.g., duration and amplitude of a camera shake, are discussed and then quantified by their influence on the perceived quality. 
An in-depth understanding of the recording quality is essential for any \ac{UGV} application.

We leverage the understanding of the recording quality for content adaptation decisions, in particular for a video composition application that is described in Chapter~\ref{chapter:600_videocomposition}.
Furthermore, video composition relies on the availability of in-parallel recorded videos from different devices at different positions.
Our understanding of video composition requires that recordings capture the same scene as different video views.
In this context, no subjectively approved quality models exist, which determine the impact of the recording position.
The second contribution of this work is a quality model describing the impact of the recording position with respect to its distance and angle in relation to a \ac{PoI}.

This chapter describes ideas, concepts and results presented in our peer-reviewed publications~\cite{Wilk2013,Wilk2014b,Wilk2014}.