% !TeX spellcheck = en_US
\section{Conclusion}
\label{sec:540_conclusion}
This chapter introduced the content- and mechanism-adaptive \ac{LiViU}, which allows for the efficient delivery of video streams to both remote and close-by receivers.
The mechanisms supporting an efficient upload are gathered in an initial simulative study investigating existing \ac{MBS} protocols.
It is shown for remote streaming, that \ac{UDP}-based protocols in combination with quick application-layer join procedures are most favorable to achieve low join times.
The initial join time is critical, as it defines a lower bound for the liveliness for the remaining streaming session.
The scheduling of the video stream messages can be either push- or pull-based.
Which scheduling should be favored depends on the context, but especially on the application.

From the findings gained in the simulative study, the novel \ac{LiViU} system has been designed.
In the remote streaming case, \ac{LiViU} achieves high bit rate video streaming with minimal overhead.
The proposed protocol includes an on-device transcoding of different video representations, and brings adaptive video streaming to the upload side. 
Content adaptation in \ac{LiViU} ensures high continuity of the streaming session without stalling.
It is supported by a scheduling mechanism adaptation, which allows \ac{LiViU} to specify which parts of which video stream shall be transmitted using push- or pull-based delivery.
This functionality is offered to the multimedia application.
This allows a high flexibility for the proposed multimedia applications in this thesis (\ac{PaSC} and video composition), but also for future developments.

For in situ streaming scenarios, \ac{LiViU} achieves a reliable streaming experience even under motion without infrastructure support.
In the in situ streaming case, \ac{LiViU} pursues a geographic distribution of the streams avoiding unnecessary redundancy in the message delivery.
As a result, with \ac{LiViU} a superior \ac{MBS} is proposed, which is used for the \ac{PaSC} proposed in the previous chapter as well as the delivery of media streams for video composition, as discussed in the next chapter.