% !TeX spellcheck = en_US
This chapter describes a novel \ac{MBS} that offers content-adaptive uploading of media streams.
Many existing \ac{MBS}s rely on the unadaptive video transmission using \ac{RTMP}, which results in the need for consistently high throughput rates.
The proposed \acf{LiViU} uses adaptive video streaming to deal with changing network conditions.
In addition, \ac{LiViU} reacts to changing application requirements and different scenarios by switching mechanisms.
The term mechanism is derived from the Future Internet project MAKI\footnote{Deutsche Forschungsgemeinschaft Collaborative Research Cluster 1053 on Multi-mechanism Adaptation in the Future Internet (MAKI)}, and specifies single or multiple protocols offering a specific function within a communication network~\cite{Gross2013}.
In this thesis, the replacement of a running mechanism with another mechanism offering a similar functionality is called mechanism adaptation~\cite{Froemmgen2015}\footnote{According to the terminology of the Future Internet project MAKI, the mechanism adaptations described in this chapter can also be termed as a transition between mechanisms.}.
Mechanism adaptations are necessary, as \ac{LiViU} has not only to deal with varying network conditions but also heterogeneous streaming scenarios (i.e., remote, in situ and hybrid streaming), and support for different applications.
The various applications discussed in this thesis are the \ac{PaSC}, which leverages in situ and remote devices for efficient quality assessment of media streams (see Chapter~\ref{chapter:550_scalable_quality_assessment}), and the video composition proposed in Chapter~\ref{chapter:600_videocomposition}.
By combining the content and mechanism adaptation, \ac{LiViU} better copes with unsteady, resource-capped networks and changing application requirements.

Mechanisms used in \ac{LiViU} are selected based on a study of different upload protocols.
To  select the most beneficial mechanisms, simulative studies with state-of-the-art \ac{MBS}s are performed.
It is shown that existing mechanisms have different strengths as well as weaknesses depending on the environmental conditions.
The strengths of different mechanisms influence the design of \ac{LiViU}, that can dynamically adapt between mechanisms to not only allow a high bit rate, but also a low delay in streaming.
A prototype on the Android \ac{OS} is proposed and evaluated, which combines mechanism and content adaptation.

Concepts and ideas discussed in this chapter revise the peer-reviewed publications~\cite{Stohr2016,Wilk2014c,Wilk2016f,Wilk2015d}.