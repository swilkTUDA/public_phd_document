% !TeX spellcheck = en_US
\section{Conclusion}
\label{sec:557_Conclusion}
An essential step for many multimedia applications is assessing the quality of video streams.
This chapter discusses a scalable and adaptive quality assessment module. 
It can run arbitrary quality assessment algorithms on mobile devices or servers in a way, that multimedia applications can set requirements for the execution.
The \ac{PaSC} is a module that selects from a set of algorithms the best one according to specified requirements.
The selected algorithm is placed on a processing device which ensures in-time completion and fulfillment of application requirements.
\ac{PaSC} offers a scalable solution to achieve the selection of the best algorithm and device at a given time to perform a quality assessment.
While inspecting the different quality assessment algorithms, a significant lack of algorithms for the detection and assessment of degradations was identified.
\ac{PaSC} offers a set of algorithms to detect and assess the impact of the degradations: camera shaking, camera misalignment, and harmful occlusions.
A set of algorithms was proposed, which rely on either a single sensor input (i.e., a camera) or a combination with other auxiliary sensors, such as an accelerometer or a gyroscope. The proposed algorithms show superior performance and reduced runtime - and they are the first to quantify the decrease in quality of recording degradations. 
