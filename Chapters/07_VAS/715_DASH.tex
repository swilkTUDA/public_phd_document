% !TeX spellcheck = en_US
\section{Recap on HTTP Adaptive Streaming (HAS)}
\label{sec:715_has}
The proposed \ac{VAS} system is designed to support the predominant video streaming protocol reaching peaks of up to 29.7\% of United States of America's downlink traffic: \acf{HAS}~\cite{Adhikari2012}.
Protocols, such as \ac{HLS} or \ac{MPEG} \ac{DASH}, belong to \ac{HAS} and have in common that the same video is available in different quality version.
Clients can switch between the different quality version during a streaming session at well-defined adaptation points.
This is possible as \ac{HAS} systems rely on \ac{HTTP} for streaming video.
Thus, streaming clients regularly request chunks of a video stream via \ac{HTTP} GET requests and get as payload of the server's response the video data.
Thus, all \ac{HAS} protocols rely on the reliable and ordered \ac{TCP}-based transmission.
In addition, \ac{HAS} clients have in common, that they leverage a playback buffer. 
Multiple chunks of video are stored before playback begins to achieve a stable and continuous playback of the video streams.
%Protocol developers can choose the desired size of the playback buffer. 
%The aim of nearly all protocols is to keep the playback buffer fill rate close to 100\%.
%Thus, the video playback is artificially delayed by a  seconds.

The most prominent and promising example for \ac{HAS} is the \ac{MPEG} standard \ac{DASH}. %, as it is vendor-independent and supported by major industry players such as Akamai, Google or Netflix.
\ac{DASH}~\cite{Stockhammer2011} standardizes the protocols for network communication of an adaptive video streaming system as well as the description of the different video qualities.  
Different video representations are encoded with different target bitrates.
The bitrates of the representations are stable over the course of the video, e.g., Apple requires that the variance is below 25\% for the latest Apple TV~\cite{Apple2016}.
For compatibility with mobile devices a constant bitrate or a maximum variance of 10\% is recommended~\cite{Apple2016}.  
The representations of a video are stored independent of each other and then split into segments of equal duration.
It is assumed, that with increasing bitrate of a \ac{DASH} representation the perceived quality of the video increases.
The segment durations determine the intervals when client's can adapt the video quality.
Segment durations usually range between two and thirty seconds. 
A web server can be used for the distribution of the video segments, as the clients pull them using \ac{HTTP}. 
\ac{DASH} uses a manifest for describing segments of the video and different quality representations. 
The manifest eases the selection of the appropriate representation by including meta data such as the resolution, bitrate and frame rate of the related video. 
Depending on the network conditions the client decides at runtime to increase or decrease quality by selecting the next segment accordingly. 
%Available adaptation schemes can be classified into throughput-based, buffer-based and hybrid ones.
%Throughput-based adaptation schemes the representation is chosen so that its bitrate is close but below the current application layer throughput.

Bitrates of a representation are affected by the video resolution, the signal-to-noise level (or the target quantization) and the frame rate of the video. 
In most cases, DASH clients rely on adaptation logics which consider the application layer throughput, the current buffer fill rate or combine both metrics for selecting the representation to be streamed~\cite{Thang2014}.
%Recent research~\cite{Huang2014} has shown that playback buffer-based or hybrid method perform best and avoid too frequent oscillations, which can become degrading for the watching experience. 
%Note, that most existing implementations neglect different frame rates, as players are not capable to switch during the run-time~\footnote{Examples of implementations which do \emph{not} support frame rate adaptation include the libDASH~\url{https://github.com/bitmovin/libdash}, JWPlayer~\url{https://github.com/jwplayer/jwplayer} or Shaka-Player~\url{https://github.com/google/shaka-player}. For our experiments we leverage the DASH IF reference implementation client DASH.js~\url{https://github.com/Dash-Industry-Forum/dash.js/}, which offer frame rate adaptation.}.

That the video content can help to adapt in a data traffic saving manner is neglected by current \ac{HAS} systems.
The proposed \ac{VAS} introduces content-aware adaptation for \ac{VAS}, which saves data and improves the video streaming experience. 