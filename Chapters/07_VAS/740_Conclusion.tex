% !TeX spellcheck = en_US
\section{Conclusion}
This chapter introduced \ac{VAS}, a system to support video adaptation on mobile video streaming clients in a content- and quality-aware manner.
The rationale behind \ac{VAS} is that current \ac{MPEG} \ac{DASH} clients do not investigate the content encoded in a video. % or the capabilities of a mobile streaming device such as supported resolutions and frame rates.
Especially in situations when a network has sufficient throughput to stream the highest representation, huge data savings are possible when \ac{VAS} is used.
\ac{VAS} leverages image processing algorithms and an objective video quality metric to understand what is encoded within a video and how it is perceived by humans.
%For the service a real-time capable video quality assessment module is proposed called \ac{SUQA}, which allows the parallel estimation of perceived quality of several \ac{DASH} representation below real-time.
\ac{VAS} analyzes the video content on its structural, temporal and color characteristics to classify segments of videos with similar characteristics. 
It was shown that quality models can be mapped between segments of different video sequences without significant loss of precision.
Both concepts allow real-time calculation of live video streams and scalability to a multitude of different video streams, which was demonstrated in a real deployment~\cite{Wilk2016c}.

\ac{VAS} allows clients to express the desired quality regarding the \ac{MOS} and links it to \ac{MPEG} \ac{DASH} representations.
Besides an optimal adaptation scheme, 
two adaptation support heuristics are introduced which support \ac{MPEG} \ac{DASH} adaptation schemes: \ac{TQA} and \ac{SQA}.
The \ac{TQA} allows to clients stream a specific target quality level, whereas classical adaptation schemes solely rely on bit rates.
It aims for saving data traffic by not selecting \ac{MPEG} \ac{DASH} representations that surpass the desired quality.
\ac{SQA} adds an adaptation support method, which mitigates adaptation effects introduced by \ac{TQA} and \ac{MPEG} \ac{DASH} adaptation schemes.
\ac{SQA} leverages the knowledge of the video content to execute adaptations in a covert manner.
The evaluation indicates that significant data traffic savings can be achieved (up to 82.83\%) without any decrease in quality.
% and simultaneously achieve small, but significant quality improvements.
%The introduction of \ac{SQA} furthermore allows efficient mitigation of any additional degradations induced by adaptations. %, which was not possible before.