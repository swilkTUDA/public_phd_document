% !TeX spellcheck = en_US
\section{Concept of VAS}
\label{sec:720_DesignPrinciples}
\subsection{Goals of VAS}
The vision for \ac{VAS} is to enable high-quality streaming sessions with a minimum of generated data traffic.
\ac{VAS} is designed for \ac{HAS} clients and focuses on mobile playback clients. 
In the remaining work, it is assumed that \ac{VAS} leverages the video streaming protocol \ac{MPEG} \ac{DASH}. 
However, the proposed concepts can easily be mapped to other \ac{HAS} protocols such as \ac{HLS}~\cite{hlsDraft} or Adobe \ac{HTTP} Dynamic Streaming\footnote{http://www.adobe.com/de/products/hds-dynamic-streaming.html; Visited on: 10/06/2016}.
Three subgoals are derived for \ac{VAS}.

First, \ac{VAS} aims to ensure a stable \emph{perceived quality} during the entire video streaming session.
Each streaming client can specify this desired perceived quality.
\ac{VAS} components are responsible for analyzing the video to determine the perceived quality of different representations and ensure that the video is played back at the desired quality level.
If the content changes, e.g., after a shot switch, the perceived quality of a video representation may also change.
\ac{VAS} stabilizes the perceived quality and not the bit rate of a streamed video.

Stabilizing quality may introduce additional adaptations.
The second goal of \ac{VAS} is to perform adaptations between different video representations in an imperceptible manner.
If adaptations between representations are performed too abruptly, the perceived quality may be additionally degraded~\cite{Moorthy2012,Zink2003}.

Third, \ac{VAS} aims at reducing data traffic by selecting the \ac{MPEG} \ac{DASH} representation that offers the desired perceived quality at a minimal bit rate. 
Users benefit from a reduction of monetary expenses in data-capped cellular networks.
Data traffic in cellular networks is rather expensive compared to fixed network contracts, and it is often capped.
For example an unlimited data traffic contract for \ac{LTE} access costs approximately 159 Euro per month\footnote{https://www.t-mobile.de/tarifoptionen/datenoptionen; Visited on: 06/09/2016.}.
\subsection{Design Principles}
To achieve the aforementioned goals, \ac{VAS} follows five design principles (DP):
\begin{DesignPrinciples}
	\item \ac{VAS} supports clients in adapting video. \ac{VAS} is not executing adaptations on its own and complies with the \ac{MPEG} \ac{DASH} standard.
	\item \ac{VAS} requires only minimal client modifications to allow its quick adoption.
	\item \ac{VAS} recommends adaptations based on the perceived quality of the video and not the bit rate.
	\item \ac{VAS} uses image processing algorithms to estimate the perceived quality.  
	This is a computationally intensive task that should not be performed on the mobile devices.
	\item \ac{VAS}'s adaptation strategies should integrate existing research on video quality and adaptation effects on mobile devices.
\end{DesignPrinciples}

\ac{VAS} is designed to \emph{support} video adaptation on mobile clients (DP1).
Thus, the \ac{VAS} server acts as a quality assessment proxy for a mobile device that wants to adaptively stream video.
Videos are directly streamed from the video server to the client, as any additional streaming delay should be avoided - which would be introduced by content inspection and video quality estimation in the path. 
As a support service, one core principle of \ac{VAS} is to allow \ac{MPEG} \ac{DASH} clients to request and execute adaptations on their own.
Thus, \ac{VAS} is not breaking the standard of \ac{MPEG} \ac{DASH} as it only offers an additional information source for making adaptation decisions~\cite{Stockhammer2011}.
The \ac{MPEG} \ac{DASH} clients need a small modification that contacts \ac{VAS} and integrates the returned data into their decision-making process (DP2).

Another reason for deploying \ac{VAS} as a support service is the large number of calculations that it requires.
The calculations are a result of \ac{VAS}'s perceived quality estimation of different video versions (DP3).
The quality is objectively estimated with a high reliability using image processing algorithms (DP4).

The analysis of the perceived quality enhances existing adaptation services (DP5), which usually solely respect network conditions.
Related research introduced new insights regarding how to execute adaptations (see Section~\ref{sec:250_adaptation_effects}).
The adaptations introduced by \ac{VAS} shall not degrade the viewing experience (DP5).