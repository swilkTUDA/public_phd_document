% !TeX spellcheck = en_US
\section{Summary and Outlook on Contributions}
This chapter summarizes the fundamentals and existing work for the quality assessment, the recording, uploading, and the processing and distribution of live \ac{UGV}, where a special focus lies on ensuring a quality-aware content adaption. % streaming.
The application scenario describes smart mobile devices capturing video to live broadcast the streams to nearby and remote devices. 
%The upload and transfer of the videos to receiving smart mobile devices implies 
Two forms of content adaptation are proposed: (1) adaptive video streaming, which allows a switch between video versions while keeping the same content; or
(2)~video composition which dynamically selects the appropriate content at a given time.

Both concepts require a reliable and in-time video quality assessment, which is realized by objective quality assessment algorithms.
The survey on objective quality assessment algorithms shows that existing work lacks the investigation of degradations that occur during the process of recording a video.
None of the existing algorithms are based on validated quality models.
These research gaps are addressed in Chapter~\ref{chapter:400_RecordingQuality} and Chapter~\ref{chapter:550_scalable_quality_assessment} of this thesis.
The existing algorithms for quality assessment, which focus on degradations occurring during the encoding and transmission of video streams, are either slow or imprecise.
In addition, concepts are missing for conducting video quality assessment at scale.
The reduction of the runtime of the algorithms, as well as an increased scalability, is presented in Chapter~\ref{chapter:550_scalable_quality_assessment} and Chapter~\ref{chapter:500_videoUpload}.

The provisioning of adaptive video streams from smart mobile devices is an unexplored research direction, as current protocols neglect adaptability of the system and incorporation of application requirements.
A novel \ac{MBS} is presented in Chapter~\ref{chapter:500_videoUpload}.

Also, the state-of-the-art for the two content adaptation types adaptive video streaming and video composition are discussed.
Existing video composition algorithms neglect the assessment of a video's quality, cannot leverage knowledge available from directing, and are incapable of performing real-time composition.
A quality-aware and real-time composition approach is the main contribution of Chapter~\ref{chapter:600_videocomposition}.

Finally, quality-aware video delivery by using adaptive video streaming is investigated.
It is found that the content of a video has a significant influence on its perceived quality.
Existing protocols lack a content-aware adaptation of digital video streams.
A solution to this gap is given in Chapter~\ref{sec:700_VAS}.