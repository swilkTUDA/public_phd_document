% !TeX spellcheck = en_US
\section{Content-aware Video Delivery to Mobile Devices}
\label{sec:250_background_content_aware_delivery}
Adaptive video streaming is able to cope with unpredictable network conditions and to ensure the delivery of the highest achievable video quality in each streaming session.
This section discusses state-of-the-art mechanisms for adaptive video delivery and shows that recent proposals do not address video content inspection for increased efficiency of video delivery.
\subsection{Adaptive Video Streaming}
\label{sec:250_AdaptiveVideo}
Adaptive video streaming allows the adjustment of the video source to the available network conditions, i.e., the throughput, during playback of a video~\cite{DeCicco2010}.
From a conceptual point of view, different protocols have been proposed for implementing the concept, ranging from server-driven adaptation to today's predominant client-driven adaptation. 
\subsubsection{Server-driven Adaptation}
For the server-driven adaptive video protocols, the media streaming server keeps track of the state of all connected streaming clients, so it can centrally switch the video version.
The approach allows fine-granular switching as the server has full control over the media and knows the upcoming video chunks, their bit rate, and the network throughput.
This complexity on the server is often a bottleneck when the number of streaming clients increases.
The server-driven adaptation schemes have less information on the effective throughput at the client side.
As a result, the client has to regularly inform the server of the current network conditions to effectively adapt the video stream.
At the same time, the processing requirements of servers increase with the number of clients, leading to the need for server infrastructures with high processing power. 
Usually, server-driven adaptation implies that scheduling of a video stream is push-based.
Different systems have been proposed which use server-driven adaptation, mainly on the basis of push-based streaming protocols such as \ac{RTP}, \ac{RTSP} and \ac{RTCP}~\cite{Fiandrotti2010,Liu2003,Wien2007}.
\subsubsection{Client-driven Adaptation}
The state-of-the-art approach for handling video streaming systems, which are used by large-scale user groups, is client-driven~\cite{Akhshabi2011,Sandvine2016}.
Each client receiving a media stream makes its own - usually independent - decisions on which video representation to stream.
This decision is usually based on the network conditions measured at the client.
As a result, streaming servers do not have to establish persistent connections to the clients and can avoid keeping track of the client's state.
In many cases, this allows video providers to use simple and cheap web servers.
Client-driven adaptation infers a pull-based scheduling.

As no data on the streaming session is evaluated centrally, each client adapts on its own, which may lead to individually optimal, but globally suboptimal adaptation results.
Approaches exist that establish mediating instances to gather the knowledge and optimize the streaming across clients~\cite{Thomas2016}.

Client-driven adaptation is the predominant adaptive video streaming concept, especially due to \ac{HAS} protocols.
\ac{HAS} represents streaming protocols which allow a client-driven adaptation during video streaming, where the underlying communication protocols are set.
As an application layer protocol, it leverages \ac{HTTP}; and thus, \ac{TCP} is used on the transport layer. 
It ensures reliable, in-order transfer of video stream chunks.
Artifacts in the video due to lost video chunks cannot occur.
\ac{TCP}'s slow retransmission behavior and congestion control have shown drawbacks in comparison to \ac{UDP}-based streaming systems.
Even though the congestion control of \ac{TCP} is not ideal for video streaming, many downsides can be compensated when using content adaptation.
Thus, the less efficient \ac{TCP}-based streaming has been widely adopted in the industry~\cite{Akhshabi2011}.
\ac{HAS} protocols are predominant in today's \ac{IP}-based video streaming~\cite{Akhshabi2011}.
Until June 2016, 35.2\% of the North American Internet traffic is caused by videos delivered using \ac{HAS}~\cite{Sandvine2016}.
It can be said that a majority of today's video streams are delivered by client-driven adaptive streaming systems, especially \ac{HAS}.
\subsection{Dynamic Adaptive Streaming over HTTP (DASH)}
\ac{DASH}~\cite{Stockhammer2011} is the most recent development of \ac{HAS}.
\ac{DASH} is the standardized evolution of proprietary \ac{HAS} solutions such as Microsoft Smooth Streaming~\cite{SmoothStreaming2016}, Adobe's \ac{HTTP} Dynamic Streaming\footnote{http://www.adobe.com/de/products/hds-dynamic-streaming.html; Visited on: 10/06/2016} and, it is related to Apple's \ac{HLS}~\cite{hlsDraft}.
\ac{DASH} standardizes the protocols for the transport of the video stream similar to \ac{HAS}. 
It defines the description of video versions in a manifest - the \ac{MPD}.
The different versions of a video that are used for adaptation are called \emph{representations}.
The standard is agnostic to how these representations are en- or decoded, but they should have different target bit rates representing different quality levels. 
Resulting bit rates are affected by the video resolution, the signal-to-noise level (or quantization), and the frame rate of the video. 
The video characteristics, e.g., the structural complexity and the motion, affect the resulting bit rate of a video. 
The representations of a video are split to equal duration segments and stored independent of each other.

A \ac{HTTP} server is used for distributing video segments, as clients pull the segments using \ac{HTTP}. 
The manifest eases the selection of a segment and quality as it describes each segment with the representation's resolution, bit rate, and frame rate. 
Depending on the available network resources, the client decides at runtime which quality to request by selecting the next segment accordingly.
How to come up with an adaptation decision is not specified in the standard, but intensively discussed in research.
\subsection{Quality in \ac{DASH}}
\label{sec:250_adaptation_effects}
For determining the quality of individual \ac{DASH} representations, it is commonly agreed that regarding quality, the highest bit rate video representation is preferred over lower bit rate representations.
Thus, quality models depict the quality of a representation in relation to the highest bit rate representation~\cite{Hossfeld2014}.
Simple approaches see a linear relationship between the perceived quality and the bit rate.
Zinner et al. studied the relationship between the video clip quality and the bit rate, which approximately follows a logarithmic function~\cite{Zinner2010}. 
Besides subjective evaluations, objective \ac{FR} quality metrics are used to determine the perceived quality of each video representation in relation to the highest bit rate representation.
Objective quality models can be generated by objective quality assessment metrics, which are discussed in Section~\ref{sec:210_qualitymetrics}.

The \ac{TCP}-based delivery in \ac{DASH} ensures that only a limited set of degradations can occur in a video streaming session.
Severe impact on the perceived quality was shown for video playback freezes (i.e., video \ac{stalling}),  initial playback delay, and effects of adaptations between different \ac{DASH} representations.
\subsubsection{Initial Startup Delay and Video Stalling}
\label{sec:250_stalling_video}
Video streaming clients require a playback buffer, which stores received video segments before playback. 
The buffer is used to compensate throughput changes, which can lead to video \ac{stalling}.
Stalling occurs when the video playback buffer of a client depletes as the network throughput rate falls below the bit rate of the current video representation for a longer duration.
When a streaming session starts, the client has to fill the buffer to begin playback.
This initial duration is called the \emph{initial startup delay}.
It is affected by the network throughput, the streamed representation, and the buffer size.
The impact of this degradation has been discussed extensively in current research. 
It is commonly agreed that for \ac{HAS}, \ac{stalling} is the most severe quality impairment, and the initial startup delay is of lesser importance~\cite{Hossfeld2011,Mok2011,Seufert2015}.

The work of Pastrana-Vidall~\cite{PastranaVidal2004} is one of the first contributions showing that \ac{stalling} events - especially the frequency and duration of \ac{stalling} - degrade the perceived video quality.
Their finding is that a single \ac{stalling} with a long duration is preferred in comparison to multiple \acp{stalling} of shorter duration.
Additionally, video viewers prefer a periodically occurring \ac{stalling} pattern with stable intervals, in contrast to unpredictable \ac{stalling} patterns.
This finding is supported by Moorthy et al.~\cite{Moorthy2012}.

For HTTP-based streaming Mok et al. discusses that the \ac{stalling} rate is the main cause for a reduction in quality~\cite{Mok2011}. 
Their model:
\begin{equation}
MOS = 4.23 - 0.0672 * L_{ti} - 0.742 * L_{fr} - 0.106 * L_{tr} 
\end{equation}
where $L_{ti} $ (L1: 0-1 seconds; L2: 1-5 seconds; L3: more than 5 seconds) is the level of the initial starting delay, $L_{fr}$ is the level of frequency of stalling events (L1: 0-0.02; L2: 0.02 - 0.15; L3: more than 0.15) and $L_{tr} $ is the level of duration of the stalling time (L1: 0-5 seconds; L2: 5 - 10 seconds; L3: more than 10 seconds).
They normalize the different metrics to levels in a range from 0 to 3.

A similar model is proposed by Hossfeld et al.~\cite{Hossfeld2013}.
It differs in terms of the coefficients but has in common with Mok et al.'s or Van Kester et al.'s model~\cite{VanKester2011}, that the frequency of \acp{stalling} has the highest impact on the perceived quality followed by the duration.
The models show a high correlation with subjective studies in \ac{HAS} protocols.
For mobile devices such as tablets, stalling is identified as the major degradation by Floris and Atzori et al.~\cite{Atzori2014,Floris2012}.
Moorthy et al.~\cite{Moorthy2012} extend their work, indicating that the common assumption that video stalling should be avoided at any cost is not always true. 
Video adaptations to low bit rates can have a more severe impact on the perceived quality in comparison to a single stalling event.
\subsubsection{Video Adaptation} 
\label{sec:250_videoAdaptation}   
A central task of adaptation schemes in adaptive video streaming systems is to plan representation switches in a way to not distract the viewers, or decrease their viewing experience~\cite{Papadimitriou2007}.

\paragraph{Influence of Video Dimension}
We define that current video encoding standards can be adapted in the spatial (resolution), the temporal (frame rate), and the quality \ac{SNR} - or quantization - dimension.
Videos can be encoded so that adaptation is possible in each of the dimensions.
In an extensive literature review, Seufert et al. show that video dimensions should be considered in an adaptation process~\cite{Seufert2015}. 

Zinner et al.~\cite{Zinner2010} studied the effects of adaptations in the temporal and spatial dimension of a video
using objective video quality metrics.
The results show that higher resolutions should be favored rather than an increased frame rate. 
Adaptations investigated include switches in the video resolution and the frame rate.
The temporal dimension has a significant impact on the video quality when the motion in a video is high~\cite{Ghinea1998}.
Also, the \ac{SNR} dimension value can be estimated by using the bit rate as an indicator of the same content, resolution, and frame rate~\cite{Garcia2010,Zhai2008}.

Resolution or spatial adaptation is the key dimension for small screens, and the impact of an adaptation is related to the respective shot type~\cite{Knoche2007,Knoche2005}.

Toni et al. discuss how to perform encoding with an optimized set of video representations~\cite{Toni2015}. 
The results of the study show that up-sampled lower resolution videos can provide the same perceived quality in comparison with higher resolution videos; however, only certain video genres benefit from this finding.

In subjective studies by Zhai et al., it is shown that those statements cannot be generalized, as it is very dependent on the content of a video~\cite{Zhai2008}.
High-motion videos prefer a temporal adaptation, whereas others prefer adaptations in the other dimensions.
 
\paragraph{Impact of the Adaptation} 
Zink et al. discuss that the adaptation process influences the perceived quality~\cite{Zink2003}.
The core findings of Zink et al.'s work include that the frequency and amplitude of an adaptation have an effect on the perceived quality.
The amplitude defines the number of representations between the currently played back representation and the target one.
Frequent adaptations can result in a reduced overall quality in comparison to the playback of the lowest available video representation.
The amplitude of an adaptation should be kept as small as possible to avoid quality-degrading effects.

Garcia et al. analyze the effects of adaptation strategies on the perceived quality~\cite{Garcia2014}. 
Studies of different types of quality switching such as encoding, spatial, temporal, and audio switches are compared. 
They show that multiple gradual quality switches are preferred in comparison to abrupt variations. 
Frequent switches are an impedance to a good user experience, unless they allow watching the highest quality for a certain time. 
Nonetheless, a consistent quality level is generally preferred to variable quality. 

In most situations, the best adaptation depends on the encoded video content~\cite{Knoche2005,Lee2011,Rajendran2002,Wang2003,Zhai2008}.
When different video dimensions are analyzed for a content-aware adaptation, it leads to a higher perceived quality than using a single quality dimension~\cite{VanDenEnde2007}.

Moorthy et al. highlight that if quality levels exhibit a small degree of separation the adaptation cannot be perceived by users; thus, adaptations can be performed in a seamless manner~\cite{Moorthy2012}.
A similar but generalized finding is made by Ni et al.~\cite{Ni2011}.
Viewers accept quality switches of up to four quantization steps for the quality dimension, a third of the original frame rate for the temporal dimension, and only half of the original frame resolution for the spatial dimension.

Also, Moorthy et al. and Ni et al. indicate a relationship between the frequency and the amplitude of adaptations~\cite{Moorthy2012,Ni2011}. 
For example, low-frequency adaptations can reduce the perceived impact if strong quality variations occur.
More frequent adaptations are allowed if this allows a viewer to watch the higher video quality layer for at least one-third of the overall video duration.
Both research groups indicate that a constant perceived quality is preferred in comparison to highly varying, perceivable changes.

Recently, studies show that viewers get used to adaptations in a video.
A study from 2016 shows that users of a \ac{HAS} stream are no longer impaired by the number of video representation switches~\cite{Nam2016}.
In contrast, it is more important that the adaptation is conducted in a covert manner.
Their finding is that the high amplitude adaptations should be avoided.
\subsection{Existing Adaptive Streaming Systems}
In the existing literature and practice, it is shown that the distribution of digital video over the Internet is driven by the demand of high bit rate content, delivered via \ac{HAS}-based systems, which encounter the challenge that performing adaptations may have an impact on the perceived quality.
The content of a video has repeatedly been reported to have a significant impact on the perception of adaptations and the quality of video representations~\cite{Garcia2010,Ghinea1998,Knoche2007,Knoche2005,Zhai2008}, but no generalized rules have been reported.

Another challenge comes with the rise of mobile video streaming, leading to a clash with these requirements, as access contracts limit the availability of high-speed Internet in particular on mobile devices.
The leading mobile telecommunications provider in Germany recently announced that unlimited data traffic for \ac{LTE} access costs approximately 159 Euro per month\footnote{https://www.t-mobile.de/tarifoptionen/datenoptionen; Visited on: 06/09/2016.}.
\subsubsection{Categorizing Adaptive Systems}
This thesis offers a discussion of the state-of-the-art adaptive video streaming systems following certain assessment characteristics, presented here.

\emph{Mobile device (MD)} support represents if the system design of an application considers the limitations of today's mobile devices such as reduced processing capabilities, lack of codec support, and energy limitations.
Current streaming systems should cope with varying \emph{network conditions (NW)}, as they affect the streaming experience. 
This can be achieved by \emph{adapting} between different bit rate representations of the same video.
The adaptation of the video is not limited to solely the bit rate but can adapt in the different video dimensions including temporal, spatial and \ac{SNR} dimension.

As the predominant approach for the delivery of video, it shall be investigated whether the proposed systems can be mapped to or use \ac{HAS} for media delivery - and if they infer additions or modifications of the principles of \ac{HAS}, such as client-driven adaptation and the usage of \ac{HTTP}.

Adaptive streaming systems can leverage the advantages of novel video \emph{encodings}, which can be classified into \ac{SVLC} and \ac{MVLC}.
As a result, the system can leverage essential properties of the respective encoding or be agnostic to it.

Users demand quality-aware streaming, which offers the advantage of delivering the desired quality with minimal data traffic. 
\emph{Quality Metric} describes if and which objective quality metric is used for estimating the video quality.

The computational processing needs, which are implied by a quality-aware video streaming, can be enormous. 
The usage of objective quality metrics, the application of \ac{MVLC} and the adaptation considering different video dimensions may lead to scenarios in which significant preprocessing of the video content is required.
Under these circumstances, it has to be assessed if the related systems support \emph{live streaming} with currently available technology.

Finally, the two main performance criteria for mobile live streaming shall be assessed:
1) Do the approaches focus on \emph{quality-aware} streaming? 2) Do the applications consider the reduction of data traffic for mobile streaming clients? 
\subsubsection{Discussion of Related Approaches}
Literature and practice propose a set of adaptive video streaming systems, which improve video streaming in fixed and mobile networks. 
Systems aiming at quality-aware streaming are discussed in Table~\ref{tab:250_Related_Work}.
\begin{table}
\centering
\caption[Comparison of existing content-adaptive video delivery systems]{Overview of related work for content-adaptive video delivery. Features used for comparison include -- MD: capable for mobile devices; NW: respects network conditions; Content Dimension: which quality dimensions of a video are respected (B: bit rate only; T: temporal; Q: quantization/ SNR; S: spatial); HAS: HTTP Adaptive Streaming; Coding: Leveraging of SVLC anf MVLC;  Quality Metric: Used video quality metric; Live: Live streaming support; Focus - Quality: Quality aware streaming; Traffic: Data traffic reduction. +: implemented; $\circ$: compatible; - unsupported.}
\begin{tabular}{lccccccccc}
\toprule[1.5pt]
& & & & & & & & \multicolumn{2}{c}{Focus}    \\ \cline{9-10} 
 &  \textbf{MD} & \textbf{NW}  & \textbf{\specialcell{Adaptation\\ (Dimensions)}} & \textbf{HAS}  & \textbf{Encoding}   & \textbf{\specialcell{Quality\\ Metric}} & \textbf{Live} & \textbf{Quality} & \textbf{Traffic} \\ 
\toprule[1.5pt]
SARA~\cite{Juluri2015} & $\circ$ & + & B & + & SVLC  & - & $\circ$ & $\circ$ & -\\
AGBR~\cite{DeVleeschauwer2013} & +& +  & B & + & SVLC  & - & + & $\circ$ & - \\
QDASH~\cite{Mok2012} & $\circ$& +  & B & + & SVLC & $\circ$ & + & $\circ$ & -\\
PANDA~\cite{Li2014} &$\circ$ & +  & B&+ & SVLC & - & + & $\circ$ &- \\
QFAS~\cite{Cicalo2014} & $\circ$ & +  & B&+ & SVLC & SSIM& - & + & - \\
AMES~\cite{Wang2013} &+ & +  & B& - (push) & MVLC & - & $\circ$ & $\circ$ & - \\
SVC over RTP~\cite{Fiandrotti2010} & - & + & T/S & - (RTP) & MVLC & PSNR & + & + & - \\
CASV~\cite{Akyol2007} & - & +  & T/Q & - (push) & MVLC & custom & - & + & -\\
Themis~\cite{Medjiah2014}& - & +  & B & - (P2P) & MVLC & -& - & +& -\\
Transit~\cite{Wichtlhuber2014} & - & +  & T/S & - (P2P) & MVLC  & VQM & $\circ$ & + & - \\
SVC over DASH~\cite{Hossfeld2015a}& $\circ$ & + & S & + & MVLC & SSIM& - & + & - \\
QoE Proxy~\cite{Essaili2013} & + & + & B  & + &  agnostic  & PSNR & - & + & -\\
QoE HAS~\cite{Devlic2015} & + & - & S/B & $\circ$ &  agnostic  & VQM & - & + & $\circ$ \\ 
%\midrule[0.75pt]
%VAS &+ & +  & S/T/Q & + & agnostic  & VQM \\
\bottomrule[1.5pt]
\end{tabular} 
\label{tab:250_Related_Work}
 \end{table}
Simple streaming systems leverage information about the bit rate to improve the perceived quality of a streaming session.
Juluri et al. introduce \ac{SARA}, an approach that integrates detailed information on video segments in the \ac{MPD} to predict the time required for the next segment to be streamed~\cite{Juluri2015}. 
The influence of the video on the perceived quality is approximated by the relationship between the bit rates of the played back video at its highest bit rate representation.
The preprocessing step adds an additional delay, which can limit the scalability of the approach in live streaming scenarios.

These approaches do not investigate the influence of streaming over mobile networks to mobile devices, as, e.g., \ac{AGBR} does~\cite{DeVleeschauwer2013}.
\ac{AGBR} proposes an optimal scheduler in cellular networks, which is run on the cell towers and optimizes the utilization of the available throughput.
The system achieves an optimal allocation when a minimum tolerable throughput is available, and indicates a level when higher representations do not offer additional quality gains.
Quality in this context is simplified to bit rates, too, but the streaming experience is improved by avoiding frequent video adaptations.
The effects of adaptations are integrated into a \ac{HAS} adaptation scheme by Mok et al.~\cite{Mok2012}. 
The proposed \ac{QDASH} system ensures that the switches are wisely planned integrating jumps to intermediate bit rate levels between representations.
In their subjective studies, it was shown that these switches are beneficial.
Furthermore, \ac{QDASH} assumes a network probing proxy, which helps to better determine the available network throughput.
Similarly, Li et al.~\cite{Li2014} assume such a central component in the network for their \ac{PANDA} system, which optimizes the streaming across different clients by ensuring a consistent quality.
 
\ac{QFAS} pursues the same goal and proposes a system which addresses cellular network delivery; and thus, mobile streaming clients, and extends quality awareness from purely considering bit rates to applying objective quality assessment metrics~\cite{Cicalo2014}.
Simple and quick, but also slow, and precise objective quality metrics such as \ac{SSIM} are used to determine the effect of different video representations on the perceived quality.
The usage of \ac{SSIM} limits the applicability of the system to non-live streaming scenarios.

AMES is an example of a range of systems that leverage \ac{MVLC} in combination with \ac{HAS}-incompatible protocols, as they either rely on push-based delivery or \ac{P2P} assisted streaming~\cite{Fiandrotti2010,Medjiah2014,Wang2013,Wichtlhuber2014}.
AMES supports mobile devices by an efficient transcoding of \ac{MVLC} to \ac{SVLC} and can therefore adapt while transcoding.
The client-driven adaptation is no longer supported in such a delivery scheme.
In \ac{P2P}-assisted streaming, especially the approaches of Transit~\cite{Wichtlhuber2014} and P2PStream~\cite{Abboud2012} are mentioned, which leverage a sophisticated perceptual video quality metric and \ac{MVLC} for video streaming.

Hossfeld et al. combine \ac{HAS} and \ac{MVLC} by modeling a \ac{MILP} and thus an optimal quality adaptation scheme~\cite{Hossfeld2015a}.
The leveraged quality models are supported by the perceptual quality metric \ac{SSIM} and are backed by a crowdsourcing investigation on the perceived quality of the video streaming sessions.
Yielding the optimal solution by having global knowledge of a streaming session of an NP-hard optimization problem.
The approach is not applicable to be used in practice.
It is the boundary of what a heuristic could theoretically achieve. 
The proposed models focus on fairness in streaming to multiple users and the effect of stalling.
Also, the number of quality switches is modeled as an impact factor on the perceived quality, which has recently shown to be controllable, as long as the quality delta is small~\cite{Nam2016}.

The approach of Essaili et al. fulfills a majority of the proposed characteristics, as it leverages video quality metrics for supporting mobile \ac{HAS} clients~\cite{Essaili2013}. 
Both leverage a proxy to offload the task of quality assessment from mobile devices and are agnostic to media encoding. 
Essaili et al. leverage a network monitoring proxy for rewriting \ac{HTTP} requests of clients and shapes the traffic between the client and the server~\cite{Essaili2013}.
Standard \ac{HAS} clients are used and centrally coordinated by the \ac{LTE} base station.
They are controlled so that they adapt in an optimal manner to achieve a fair distribution of network resources.

Devlic et al. propose a delivery model for video streams by optimizing video content for a target quality~\cite{Devlic2015}.
It is assumed that the video content affects the perceived quality of a video representation in a long-running video sequence.
A video optimization scheme analyzes the video sequences offline, making the approach unsuitable for live streaming scenarios.
The perceived quality is estimated by using the \ac{VQM}, being highly precise but computationally intensive.
An optimization addresses the video content showing data savings are possible - yet it lacks, as an offline process, the consideration of network variations.
\subsection{Discussion}
What is obvious from Table~\ref{tab:250_Related_Work} and its discussion are the lack of applying quality-aware streaming that considers the video content.
Video content affects the perceived quality in a manner where understanding helps mobile devices to stream video at minimal data traffic.
The existing approaches focus on either ensuring that a streaming session is optimized regarding its individual perceived quality, or regarding fair share of quality across streaming receivers.
These streaming systems thereby focus on network conditions alone, trying to avoid stalling effects.
Moreover, they assume that the highest bit rate representation of a video is always beneficial for users.
Video as a medium is not addressed at all - which offers a huge potential regarding a quality-aware adaptation, as different studies show a strong connection between the content being distributed and its potential for quality adaptation.
Advanced adaptation schemes are used by \ac{P2P}-assisted streaming systems, which leverage quality as a good that can be shared with other clients.
The approaches investigate the content being streamed using recent objective quality metrics, addressing different characteristics of a video.
Those approaches are rather non-beneficial for mobile devices in cellular networks. 
Currently, no content-aware \ac{HAS} adaptation system addresses the needs of mobile clients in cellular networks.